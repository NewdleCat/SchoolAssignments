% --------------------------------------------------------------
% This is all preamble stuff that you don't have to worry about.
% Head down to where it says "Start here"
% --------------------------------------------------------------

\documentclass[12pt]{article}

\usepackage[margin=1in]{geometry}
\usepackage{amsmath,amsthm,amssymb}
\usepackage{enumerate}
\usepackage{graphicx}
\usepackage[english]{babel}
\usepackage[utf8x]{inputenc}
\usepackage[T1]{fontenc}
\usepackage{enumitem}
\usepackage{fancyhdr}
\pagestyle{fancy}


\newcommand{\N}{\mathbb{N}}
\newcommand{\Z}{\mathbb{Z}}
\newcommand{\R}{\mathbb{R}}
\newcommand{\Q}{\mathbb{Q}}
\newcommand{\C}{\mathbb{C}}

\newenvironment{theorem}[2][Theorem]{\begin{trivlist}
\item[\hskip \labelsep {\bfseries #1}\hskip \labelsep {\bfseries #2.}]}{\end{trivlist}}
\newenvironment{lemma}[2][Lemma]{\begin{trivlist}
\item[\hskip \labelsep {\bfseries #1}\hskip \labelsep {\bfseries #2.}]}{\end{trivlist}}
\newenvironment{exercise}[2][Exercise]{\begin{trivlist}
\item[\hskip \labelsep {\bfseries #1}\hskip \labelsep {\bfseries #2.}]}{\end{trivlist}}
\newenvironment{problem}[2][Problem]{\begin{trivlist}
\item[\hskip \labelsep {\bfseries #1}\hskip \labelsep {\bfseries #2.}]}{\end{trivlist}}
\newenvironment{question}[2][Question]{\begin{trivlist}
\item[\hskip \labelsep {\bfseries #1}\hskip \labelsep {\bfseries #2.}]}{\end{trivlist}}
\newenvironment{corollary}[2][Corollary]{\begin{trivlist}
\item[\hskip \labelsep {\bfseries #1}\hskip \labelsep {\bfseries #2.}]}{\end{trivlist}}

\lhead{Homework 8, August 27, 2021}
\rhead{Nicholas Tee}

\begin{document}

\subsection*{Problem 1}
\textbf{a)} Assume $|x - a| < \delta$ We want to show that $\left|\dfrac{1}{x} - \dfrac{1}{a} \right| < \epsilon$. Let us say that
\[ \left|\dfrac{1}{x} - \dfrac{1}{a} \right| = \left|\dfrac{a}{a \cdot x} - \dfrac{x}{a \cdot x} \right| = \dfrac{a - x}{ax} \]
let us take $\delta = \dfrac{a}{2}$ From here we know that $|x - a| < \dfrac{a}{2}$. This implies that $|x| > \dfrac{a}{2}$ and that $|ax| > \dfrac{a^2}{2}$ so
\[  \dfrac{x-a}{ax} < \dfrac{x - a}{\dfrac{a^2}{2}} = \dfrac{2(x - a)}{a^2}\]
From here we can say that $|x- a| < \dfrac{\epsilon a^2}{2}$ and $|ax| > \dfrac{a^2}{2}$ so
\[ \left|  \dfrac{1}{x} - \dfrac{1}{a} \right| \]
\textbf{b) } Suppose $\epsilon = 1$ and that $f(x)$ is uniformly continuous. Then for all $x,y \in (0,1]$ we have $|x-y| < 1$ and $|\dfrac{1}{x} - \dfrac{1}{y}| < 1$. We can pick some $x \in (0,1]$ and let $y= \dfrac{x}{2}$.
\begin{align*}
|x-y| &= \left |x - \dfrac{x}{2}\right |\\
&= \left |\dfrac{x}{2}\right | = \dfrac{x}{2} \\
&< \dfrac{f(x)}{2}
\end{align*}
also 
\begin{align*}
\left|\dfrac{1}{x} - \dfrac{1}{y} \right| &= \left|\dfrac{1}{x} - \dfrac{2}{x} \right|\\
&= \left |\dfrac{-1}{x}\right |\\
&= \dfrac{1}{x} > 1
\end{align*}
\subsection*{Problem 2}
For any $\epsilon > 0$ if $\lim_{x \to a}f(x) = \ell$ then it impies that $|f(x) - L| < \epsilon$. If $f(x) = \lim_{x \to a^-} = \ell$ Then for any $\delta > 0$ there there exists a $|a-x| < \delta$. Similarly if $f(x) = \lim_{x \to a^+} = \ell$ then there exists $|x - a| < \delta$. If we take $|x - a| < \delta$, it is the same if we wrote it as $-\delta < x - a < \delta$. If we flip the inequality we will get $\delta > a - x > -\delta$. Essentially this shows that both $|x-a| < \delta$ and $|a - x| < \delta$ hold, meaning that $|f(x) - \ell| < \epsilon$ will also hold.
\newpage
\subsection*{Problem 3}
\subsection*{Problem 4}
\textbf{a) } $f(x) = x^5 + x^3 + x - 1$\\
$f(0) = (0)^5 + (0)^3 + (0) - 1 = -1$\\
$f(1) = (1)^5 + (1)^3 + (1) - 1 = 2$\\
Since $f(x)$ goes from negative to positive between 0 and 1 it shows that there is a root.
\textbf{b) } If we take the limit $lim_{x \to \infty}\dfrac{1}{1+x^2} = 0$, similarly if we take $\lim_{x \to \infty}5x^5+4x^4 = \infty$. From this we know that $f(x)$ approaches $0$ meaning that it essentially covers the entire x-axis. Furthermore we can see that $g(x)$ approaches $\infty$, which means that the graph is completely vertical. We also know that both equations are continuous. Since one is going horizontally and the other vertically they are bound to intersect.
\subsection*{Problem 5}
For all $x \in (a,b)$ we know that $f'(x) = 0$. Since f is continous and differentiable in $(a,b)$ we can use the mean value theorem which states there exists a $c \in (a,b)$ such that
\[ f(b) - f(a) = f'(c)(b-a) \]
However, we know that for all $x \in (a,b)$ that $f'(x) = 0$ so we get
\[ f(b) - f(a) = (0)(b-a) \]
\[ f(b) - f(a) = 0 \]
\[ f(b) = f(a) \]
This is true for any arbitrary interval $(a,b)$ which means that $f$ is a constant.
\end{document}