% --------------------------------------------------------------
% This is all preamble stuff that you don't have to worry about.
% Head down to where it says "Start here"
% --------------------------------------------------------------

\documentclass[12pt]{article}

\usepackage[margin=1in]{geometry}
\usepackage{amsmath,amsthm,amssymb}
\usepackage{enumerate}
\usepackage{graphicx}
\usepackage[english]{babel}
\usepackage[utf8x]{inputenc}
\usepackage[T1]{fontenc}
\usepackage{enumitem}
\usepackage{fancyhdr}
\pagestyle{fancy}


\newcommand{\N}{\mathbb{N}}
\newcommand{\Z}{\mathbb{Z}}
\newcommand{\R}{\mathbb{R}}
\newcommand{\Q}{\mathbb{Q}}
\newcommand{\C}{\mathbb{C}}

\newenvironment{theorem}[2][Theorem]{\begin{trivlist}
\item[\hskip \labelsep {\bfseries #1}\hskip \labelsep {\bfseries #2.}]}{\end{trivlist}}
\newenvironment{lemma}[2][Lemma]{\begin{trivlist}
\item[\hskip \labelsep {\bfseries #1}\hskip \labelsep {\bfseries #2.}]}{\end{trivlist}}
\newenvironment{exercise}[2][Exercise]{\begin{trivlist}
\item[\hskip \labelsep {\bfseries #1}\hskip \labelsep {\bfseries #2.}]}{\end{trivlist}}
\newenvironment{problem}[2][Problem]{\begin{trivlist}
\item[\hskip \labelsep {\bfseries #1}\hskip \labelsep {\bfseries #2.}]}{\end{trivlist}}
\newenvironment{question}[2][Question]{\begin{trivlist}
\item[\hskip \labelsep {\bfseries #1}\hskip \labelsep {\bfseries #2.}]}{\end{trivlist}}
\newenvironment{corollary}[2][Corollary]{\begin{trivlist}
\item[\hskip \labelsep {\bfseries #1}\hskip \labelsep {\bfseries #2.}]}{\end{trivlist}}

\lhead{Homework 6, August 19, 2021}
\rhead{Nicholas Tee}

\begin{document}

\subsection*{Problem 1}
If we treat $n$ as the index of the sequence. Then that means the statement
\[ \lim_{n \to \infty}a_{2n} = L = \lim_{n \to \infty}a_{2n + 1} \]
Shows that $a_n$ converges. since $a_{2n}$ Is all the even indexes of the sequence while $a_{2n+1}$ takes care of all the odd indexes in the sequence. These can be treated as both subsequences of $a_n$. If all the subsequences of $a_n$ converge, then it should be true that $a_n$ also converges.
\subsection*{Problem 2}
Assuming that $\{s_n\}$ is monotonically increasing and that $\{ s_{n_{k}} \}$ and is converges. We want to show that $|s_n - s| \leq \epsilon$ for all $n \geq N$. Let us find some $k$ such that $k \geq k - \epsilon $ and $k - \epsilon < s_{n_{k}} - s < 0$. So for any $n$ we can then say that $s_{n_{k}} \leq s_n \leq s$. Then we can say
\[ -\epsilon \leq s_{n_{k}} - s \leq s_n -s \leq 0\]
Which shows that $s_n$ also converges
\subsection*{Problem 3}
\textbf{a) } if $u_n = \dfrac{n^3}{e^n}$
\begin{align*}
\lim_{n \to \infty} \dfrac{u_{n+1}}{u_n} &=  \lim_{n \to \infty}\dfrac{(n+1)^3}{e^{n+1}} \cdot \dfrac{e^n}{n^3} \\
&= \lim_{n \to \infty} \dfrac{1}{e} \cdot \dfrac{(n+1)^3}{n^3} \\
&= \dfrac{1}{e} < 1
\end{align*}
So by the ratio test, it converges.\\
\textbf{b) } $b_n = n^{-1 - \dfrac{1}{n}} = \dfrac{1}{n^{1 + \dfrac{1}{n}}}$ and $a_n = \dfrac{1}{n^2}$\\
$a_n < b_n$ and since $a_n$ diverges we know that $b_n$ will also diverge.\\\\
\textbf{c)} I think you can show that this is divergent with the comparison test, but I am unsure what to use as a lower bound\\
\textbf{d)} $a_n = (n^{\dfrac{1}{n}} - 1)^n$
\begin{align*}
\lim_{n \to \infty} |a_n|^{\dfrac{1}{n}} &= \lim_{n \to \infty} |(n^{\dfrac{1}{n}} - 1)^n|^{\dfrac{1}{n}}\\
&= \lim_{n \to \infty} | n^{\dfrac{1}{n}} - 1| \\
&= 0
\end{align*}
This shows that the sequence is absolutely convergent, which means that it is convergent.
\subsection*{Problem 4}
\textbf{a) } If we know that $\sum^{\infty}_{n = 1}a_n$ then we can say that
\[ \sum^{\infty}_{n = 1}a^2_n = \left(  \sum^{\infty}_{n = 1}a_n \right)^2 \]
Since $a_n$ converges, we can do this, meaning that we can say $a_n^2$ also converges\\\\
\textbf{b) } Assuming $a_n$ converges we can say
\[ \dfrac{a_n}{1+ a_n} = \dfrac{a_n}{1 - a_n} = \dfrac{a_n}{1 - \dfrac{1}{2}} = 2 \cdot a_n\]
Since $a_n$ converges, using the comparison test we can say that $\dfrac{a_n}{1 + a_n}$ also converges. \\\\
\textbf{c)} Using the same logic as the last problem we can say that
\[ \dfrac{a^2_n}{1+ a^2_n} = \dfrac{a^2_n}{1 - a^2_n} = \dfrac{a^2_n}{1 - \dfrac{1}{2}} = 2 \cdot a_n^2 \]
in part a) we showed that $a_n^2$ is convergent. So through the comparison test we can once again show that $\dfrac{a^2_n}{1+ a^2_n}$ is also convergent. 
\subsection*{Problem 5}
Not necessarily. Take the sequence $a_n = \dfrac{1}{n}$. We know that this sequence will converge towards 0 if we take the limit. However, it is a different story if it is in a series. We know that $\sum_{n = 1}^{\infty} \dfrac{1}{n}$ is the harmonic series, which diverges.

\end{document}