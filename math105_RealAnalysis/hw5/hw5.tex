% --------------------------------------------------------------
% This is all preamble stuff that you don't have to worry about.
% Head down to where it says "Start here"
% --------------------------------------------------------------

\documentclass[12pt]{article}

\usepackage[margin=1in]{geometry}
\usepackage{amsmath,amsthm,amssymb}
\usepackage{enumerate}
\usepackage{graphicx}
\usepackage[english]{babel}
\usepackage[utf8x]{inputenc}
\usepackage[T1]{fontenc}
\usepackage{enumitem}
\usepackage{fancyhdr}
\pagestyle{fancy}


\newcommand{\N}{\mathbb{N}}
\newcommand{\Z}{\mathbb{Z}}
\newcommand{\R}{\mathbb{R}}
\newcommand{\Q}{\mathbb{Q}}
\newcommand{\C}{\mathbb{C}}

\newenvironment{theorem}[2][Theorem]{\begin{trivlist}
\item[\hskip \labelsep {\bfseries #1}\hskip \labelsep {\bfseries #2.}]}{\end{trivlist}}
\newenvironment{lemma}[2][Lemma]{\begin{trivlist}
\item[\hskip \labelsep {\bfseries #1}\hskip \labelsep {\bfseries #2.}]}{\end{trivlist}}
\newenvironment{exercise}[2][Exercise]{\begin{trivlist}
\item[\hskip \labelsep {\bfseries #1}\hskip \labelsep {\bfseries #2.}]}{\end{trivlist}}
\newenvironment{problem}[2][Problem]{\begin{trivlist}
\item[\hskip \labelsep {\bfseries #1}\hskip \labelsep {\bfseries #2.}]}{\end{trivlist}}
\newenvironment{question}[2][Question]{\begin{trivlist}
\item[\hskip \labelsep {\bfseries #1}\hskip \labelsep {\bfseries #2.}]}{\end{trivlist}}
\newenvironment{corollary}[2][Corollary]{\begin{trivlist}
\item[\hskip \labelsep {\bfseries #1}\hskip \labelsep {\bfseries #2.}]}{\end{trivlist}}

\lhead{Homework 5, August 15, 2021}
\rhead{Nicholas Tee}

\begin{document}

\subsection*{Problem 1}
\textbf{a) }
\begin{proof}
base case: $n = 1$
\[ 9(10^{n-1} + 10 ^{n-2} + ... + 10 + 1) \leq 10^n \]
\[ 9(10^{1-1}) \leq 10^1 \]
\[ 9 \leq 10 \]
Inductive step: assume $9(10^{n-1} + 10 ^{n-2} + ... + 10 + 1) \leq 10^n$\\
Prove that $10^{n+1} \geq 9(10^{n} + 10^{n-1} + 10^{n-2} + ... + 10 + 1)$, using the assumption we can say that
\begin{align*}
10^{n} \cdot 10 &\geq 9(10^{n-1} + 10 ^{n-2} + ... + 10 + 1) \cdot 10 \\
&= 9(10^n + 10^{n-1} + 10^{n-2} + ... + 100 + 10)\\
&\geq 9(10^n + 10^{n-1} + 10^{n-2} + ... + 10 + 1)
\end{align*}
This shows that $9(10^{n-1} + 10 ^{n-2} + ... + 10 + 1) \leq 10^n$ for all $n \in \N$ \\
\end{proof}
\textbf{b)} Prove that 
\[ \dfrac{9}{10^{m+1}} + \dfrac{9}{10^{m+2}} + ... + \dfrac{9}{10^{n}}  \leq \dfrac{1}{10^m} \]
\begin{align*}
10^m \left( \dfrac{9}{10^{m+1}} + \dfrac{9}{10^{m+2}} + ... + \dfrac{9}{10^{n}}  \right) &\leq \dfrac{10^m}{10^m} \\
\dfrac{9}{10} + \dfrac{9}{100} + ... + \dfrac{9 \cdot 10^m}{10^{n}} &\leq 1 \\
10^n \left( \dfrac{9}{10} + \dfrac{9}{100} + ... + \dfrac{9 \cdot 10^m}{10^{n}}  \right) &\leq 10^n \\
9( 10^{n-1} + 10^{n-2} + ... + 10^m ) &\leq 10^n
\end{align*}
We can then say that
\[ 9( 10^{n-1} + 10^{n-2} + ... + 10^m ) \leq  9(10^{n-1} + 10 ^{n-2} + ... + 10 + 1)\]
Since the LHS is a sum from $n$ to $m$ whereas the RHS is a sum from $n$ to $1$. From this we can conclude that 
\newpage
\subsection*{Problem 2}
\textbf{a)} Prove that $2^{n-1} + 2^{n-2} + ... + 1 \leq 2^n$
\begin{proof}
Base Case: $n = 1$
\[ 2^{1-1} \leq 2^1 \]
\[ 1 \leq 2 \]
Assume that $2^{n-1} + 2^{n-2} + ... + 1 \leq 2^n$ \\
Prove that $2^n + 2^{n-1} + 2^{n-2} + ... + 1 \leq 2^{n+1}$ \\
using the assumption we can say that
\begin{align*}
2^n \cdot 2 &\geq (2^{n-1} + 2^{n-2} + ... + 1) \cdot 2 \\
&= 2^n + 2^{n-1} + ... + 2 \\
&\geq 2^n + 2^{n-1} + ... + 1
\end{align*}
This shows that $2^{n-1} + 2^{n-2} + ... + 1 \leq 2^n$ is true for all $n \in \N$
\end{proof}
\textbf{b) } Prove that $\dfrac{1}{k!} \leq \left(  \dfrac{1}{2} \right) ^{k-1}$
\begin{proof}
Base Case: $k = 2$
\[ \dfrac{1}{2!} \leq \left( \dfrac{1}{2} \right) ^{2-1} \]
\[ \dfrac{1}{2} \leq \dfrac{1}{2} \]
Assume that $\dfrac{1}{k!} \leq \left(  \dfrac{1}{2} \right) ^{k-1}$ prove that $\dfrac{1}{(k+1)!} \leq \left(  \dfrac{1}{2} \right) ^{k}$
Using the assumption we can say that
\begin{align*}
\dfrac{1}{(k+1)!} &= \dfrac{1}{(k+1)k!} \\
&= \dfrac{1}{k+1} \cdot \dfrac{1}{k!} \\
&\leq \dfrac{1}{k+1} \cdot \left(  \dfrac{1}{2} \right)^{k-1} = \dfrac{1}{2^{k-1}(k+1)}\\
&\leq \dfrac{1}{2^k}
\end{align*}
To show that the inequality is true we need to show that 
\begin{align*}
2^{k-1}(k+1) &\geq 2^k \\
2^{k-1}(k+1) &\geq 2^{k-1} \cdot 2
\end{align*}
we know that this statement is true because since $k \geq 2$ then the minimum value of $k+1$ is 3. So $2^{k-1}(k+1) \geq 2^k$ is true which then proves that $\dfrac{1}{k!} \leq \left(  \dfrac{1}{2} \right) ^{k-1}$
\end{proof}
\newpage
\textbf{c)}
\begin{proof}
Case 1: $n = 0$. We get $s_0 = 1$\\
Case 2: $n = 1$. We get $s_1 = 1 + 1 = 2$ \\
Case 3: $n = 2$. we get $s_2 = 1 + 1 + \dfrac{1}{2} = 2.5$. We can then say that for any $n \in \N$ such that $n \geq 3$
\[ s_n = 1 + 1 + \dfrac{1}{2} + ... + \dfrac{1}{(n-1)!} + \dfrac{1}{n!} \]
In order to show that $s_n \leq 3$ all we need to show is that (we take out the base case of n = 0)
\[ 1 + \dfrac{1}{2} + ... + \dfrac{1}{(n-1)!} + \dfrac{1}{n!} \leq 2 \]
Using our result from a) we can do this
\begin{align*}
1 + \dfrac{1}{2} + ... + \dfrac{1}{(n-1)!} + \dfrac{1}{n!} &\leq 1 + \dfrac{1}{2} + ... + \dfrac{1}{2^{n-2}} + \dfrac{1}{2^{n-1}} \\
&= 2^{n-1} + \dfrac{2^{n-1}}{2} + ... + \dfrac{2^{n-1}}{2^{n-2}} + \dfrac{2^{n-1}}{2^{n-1}} \\
&= 2^{n-1} + 2^{n-2} + ... + 2 + 1 \leq 2^{n} \\
&= \dfrac{2^{n-1} + 2^{n-2} + ... + 2 + 1}{2^{n-1}} \leq \dfrac{2^n}{2^{n-1}} \\
&= 1 + \dfrac{1}{2} + ... + \dfrac{1}{2^{n-2}} + \dfrac{1}{2^{n-1}} \leq 2
\end{align*}
From this we can conclude that for all $n \in \N$ that $s_n \leq 3$\\
\end{proof}
\subsection*{Problem 3}
prove that 
\[ x = 0.a_1a_2 \cdots a_na_1a_2 \cdots a_na_1a_2 \cdots \]
for all $n \in \N$. Let us multiply it by $10^n$ so we will end up with
\[ x \cdot 10^n = a_1a_2 \cdots a_n.a_1a_2 \cdots a_na_1a_2 \cdots \]
\[ x \cdot 10^n - x = a_1a_2 \cdots a_n \]
\[x(10^n - 1) = a_1a_2 \cdots a_n\]
\[ x = \dfrac{a_1a_2 \cdots a_n}{10^n - 1} \]
This shows that x is a rational number.
\newpage
\subsection*{Problem 4}
Prove that $\{s_n\}$ converges such that $s_n = \sum^{n}_{k=0} \dfrac{a_k}{k!}$
$s_0 = a_0$\\
$s_1 = a_0 + a_1$\\
$s_2 = a_0 + a_1 + \dfrac{a_2}{2}$ \\
$s_3 = a_0 + a_1 + \dfrac{a_2}{2} + \dfrac{a_3}{6}$ \\
$s_n = a_0 + a_1 + \dfrac{a_2}{2} + \dfrac{a_3}{6} + \cdots + \dfrac{a_n}{n!}$ \\
Each iteration of the sequence $\{ s_n \}$ gets slightly larger, which would mean that the sequence is monotonically increasing. We also know that it is bounded as $\dfrac{a_k}{k!}$ approaches 0. since $a_k$ is bounded and the denominator of the function is a factorial. as it approaches infinity it will converge towards 0. Which means that the summation is bounded. Which means that the sequence is convergent.
\subsection*{Problem 5}

\end{document}