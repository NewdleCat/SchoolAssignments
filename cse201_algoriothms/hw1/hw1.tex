% --------------------------------------------------------------
% This is all preamble stuff that you don't have to worry about.
% Head down to where it says "Start here"
% --------------------------------------------------------------

\documentclass[12pt]{article}

\usepackage[margin=1in]{geometry}
\usepackage{amsmath,amsthm,amssymb}
\usepackage{enumerate}
\usepackage{graphicx}
\usepackage[english]{babel}
\usepackage[utf8x]{inputenc}
\usepackage[T1]{fontenc}
\usepackage{enumitem}
\usepackage{fancyhdr}
\pagestyle{fancy}


\newcommand{\N}{\mathbb{N}}
\newcommand{\Z}{\mathbb{Z}}
\newcommand{\R}{\mathbb{R}}
\newcommand{\Q}{\mathbb{Q}}
\newcommand{\C}{\mathbb{C}}

\newenvironment{theorem}[2][Theorem]{\begin{trivlist}
\item[\hskip \labelsep {\bfseries #1}\hskip \labelsep {\bfseries #2.}]}{\end{trivlist}}
\newenvironment{lemma}[2][Lemma]{\begin{trivlist}
\item[\hskip \labelsep {\bfseries #1}\hskip \labelsep {\bfseries #2.}]}{\end{trivlist}}
\newenvironment{exercise}[2][Exercise]{\begin{trivlist}
\item[\hskip \labelsep {\bfseries #1}\hskip \labelsep {\bfseries #2.}]}{\end{trivlist}}
\newenvironment{problem}[2][Problem]{\begin{trivlist}
\item[\hskip \labelsep {\bfseries #1}\hskip \labelsep {\bfseries #2.}]}{\end{trivlist}}
\newenvironment{question}[2][Question]{\begin{trivlist}
\item[\hskip \labelsep {\bfseries #1}\hskip \labelsep {\bfseries #2.}]}{\end{trivlist}}
\newenvironment{corollary}[2][Corollary]{\begin{trivlist}
\item[\hskip \labelsep {\bfseries #1}\hskip \labelsep {\bfseries #2.}]}{\end{trivlist}}

\lhead{Homework 1, October 9, 2022}
\rhead{}

\begin{document}

\subsection*{Problem 1}
\begin{proof}
First we can recall the definitions of both $O(g(n))$ and $\Omega(g(n))$
\[ O(g(n)) \in \{ f(n)|\exists c>0, \exists n_0>0, \forall n \geq n_0: 0 \leq f(n) \leq cg(n) \} \]
\[ \Omega (g(n)) \in \{ f(n)|\exists c>0, \exists n_0>0, \forall n \geq n_0: 0 \leq cg(n) \leq f(n) \} \]
Now if we look at the intersection of both sets, i.e $O(g(n)) \cap \Omega(g(n))$, $O(g(n))$ is the set such that all $f(n) \leq cg(n)$ and the set $\Omega(g(n))$ is the set such that all $cg(n) \leq f(n)$. Thus the intersection of both sets will be where $c_1g(n) \leq f(n) \leq c_2g(n)$. We can then formally define the set as 
\[ \Theta(g(n)) \in O(g(n)) \cap \Omega(g(n)) = \{ f(n)|\exists c>0, \exists n_0>0, \forall n \geq n_0: 0 \leq cg(n) \leq f(n) \} \]
Thus we have shown that both definitions are equivalent.
\end{proof}
\subsection*{Problem 2}
Since the log base is just a constant, I will just have them in base $e$ to make my life easier.
\textbf{a)} 
\begin{align*}
\lim_{n \rightarrow \infty} \dfrac{2^n}{(n lg(n))^2} &= \lim_{n \rightarrow \infty} \dfrac{ln(2) \cdot 2^n}{2n \cdot lg(n)(lg(n) + 1)}\\
&= \lim_{n \rightarrow \infty} \dfrac{2^n}{2lg(n)+6lg(n)+2}\\
&= \lim_{n \rightarrow \infty} \dfrac{n \cdot 2^n}{4lg(n)+6}\\
&= \lim_{n \rightarrow \infty} n(n+1)2^n = \infty
\end{align*}
Thus $f(n) = \omega(g(n))$\\\\
\textbf{b)}
\begin{align*}
\lim_{n \rightarrow \infty} \dfrac{\sqrt{n}}{lg(n)} &= \lim_{n \rightarrow \infty} \dfrac{\dfrac{1}{2}n^{-\dfrac{1}{2}}}{\dfrac{1}{n}}\\
&= \lim_{n \rightarrow \infty} \dfrac{1}{2}n^{\dfrac{1}{2}} = \infty
\end{align*}
$f(n) = \omega(g(n))$\\\\
\newpage
\textbf{c)}
\begin{align*}
\lim_{n \rightarrow \infty} \dfrac{lg(n)^{lg(n)}}{n^3} &= \left(  \lim_{n \rightarrow \infty} \dfrac{lg(n)}{n^3} \right)^{lg(n)}\\
&= \lim_{n \rightarrow \infty} \left( \dfrac{\dfrac{1}{n}}{3n^2} \right)^{lg(n)} = 0
\end{align*}
$f(n) = o(g(n))$\\\\
\textbf{d)}
$f(n) = o(g(n))$ because $3^{2^n}$ grows at a lower rate than $2^{3^n}$\\\\
\textbf{e)}
\begin{align*}
\lim_{n \rightarrow \infty} \dfrac{n!}{(n+1)!} &= \lim_{n \rightarrow \infty} \dfrac{n!}{n!(n+1)}\\
&= \lim_{n \rightarrow \infty} \dfrac{1}{n+1} = \infty
\end{align*}
$f(n) = o(g(n))$\\\\
\textbf{f)}
\begin{align*}
test
\end{align*}
\subsection*{Problem 3}
\begin{proof}
\textbf{a)} If $f(n) \in O(g(n))$, this implies that there exists some constant $c$ such that $c \in \N$ and $c > 1$, so that $f(n) \leq cg(n)$. From here we have it such that
\begin{align*}
f(n) \leq cg(n) &\rightarrow lg(f(n)) \leq lg(cg(n))\\
&= lg(f(n)) \leq lg(c) + lg(g(n))\\
&\text{since $lg(g(n)) \geq 1$ we can do the following}\\
&= lg(f(n)) \leq lg(c) \cdot lg(g(n)) + lg(g(n))\\
&= lg(f(n)) \leq (lg(c) + 1)lg(g(n))\\
&= lg(f(n)) \leq O(lg(g(n)))
\end{align*}
Thus proven that $f(n) \in O(g(n))$ implies that $lg(f(n)) \in lg(O(g(n))$\\
\end{proof}
\begin{proof}
\textbf{b)} $f(n) = O(g(n))$ does not imply that $2^{f(n)} \in O(2^{g(n)})$ This can be proven by counter example. Assume that $f(n) = 2n$ and let $g(n) = n$. We can see that $2n \in O(n)$ since
\[ \lim_{n \rightarrow \infty} \dfrac{2n}{n} = 2 \]
However, if we take the limit as such
\[ \lim_{n \rightarrow \infty} \dfrac{2^{2n}}{2^n} = \lim_{n \rightarrow \infty} 2^n = \infty\]
Which means that $2^{f(n) }  \not \in O(2^{g(n)})$\\
\end{proof}
\subsection*{Problem 4}
\begin{proof}
Let $Q(N)$ be some with degree $k$ such that $k \in \N$ and $k >0$. We can say that
\[ Q(N) = c_1x^{k} + c_2x^{k-1} + ... + c_n \]
where $c_1, c_2,..., c_n$ are coefficients. Similarly, we can say that $P(N)$ is another polynomial with degree $n$ such that $n \in \N$ and $0<n\leq k$, we can represent $P(N)$ as such
\[ P(N) = d_1x^{n} + d_2x^{n -1} + ... + d_n \]
where $d_1, d_2, ..., d_n$ are a different set of coefficients. We can then take the limit of $\dfrac{P(n)}{Q(n)}$ as such
\[ \lim_{n \rightarrow \infty} \dfrac{P(n)}{Q(n)} = \lim_{n \rightarrow \infty} \dfrac{c_1x^{k} + c_2x^{k-1} + ... + c_n}{d_1x^{n} + d_2x^{n -1} + ... + d_n} \]
If we take this limit, we know that both $P(n)$ and $Q(n)$ go to $\infty$. So we can use L'hoptials rule and derive both functions. Since, its a polynomial, the functions will keep deriving, until the coefficient of the highest degree term will be left (i.e $c_1$ or $d_1$).\\
In case that $n < k$, then the limit will be reduce to the following
\[ \lim_{n \rightarrow \infty} \dfrac{0}{d_1} = 0 \]
In the second case where $k = n$, then the limit will be reduced as such
\[ \lim_{n \rightarrow \infty} \dfrac{c_1}{d_1} \]
If $ \lim_{n \rightarrow \infty} \dfrac{P(n)}{Q(n)} = L$. Then we can say that $0\leq L < \infty$. So $P(n) \in O(Q(n))$\\
\end{proof}
\newpage
\subsection*{Problem 5}
\textbf{Case 1: a = 1} If $a = 1$ then the expression $f(n)= \Sigma^{n}_{i=1}a^i = n$. Since 1 to the power of anything will simply become 1, so the expression becomes a summation of 1, n n number of times. With this, we can say that $g(n) = n$, we then have 
\[ \lim_{n \rightarrow \infty}\dfrac{f(n)}{g(n)} = \lim_{n \rightarrow \infty} \dfrac{n}{n} =\lim_{n \rightarrow \infty} 1 = 1\]
With this we can say that if a = 1, then $f(n) \in \Theta (n)$\\\\
\textbf{Case 2: a > 1} The closed form for the summation above is essentially just the geometric series
\[ \Sigma^{n}_{i=1} a^i = \dfrac{a^{n+1} - 1}{a - 1} = \dfrac{a \cdot a^n - 1}{a - 1} = \dfrac{a^{n+1}}{a-1} - \dfrac{1}{a-1} \]
We know that $\forall n ,\Sigma^{n}_{i=1} a^i < \dfrac{a^{n+1}}{a-1}$ and $\Sigma^{n}_{i=1} a^i > a^n$, so
\begin{align*}
c^n &< \dfrac{a^{n+1}}{a-1} - \dfrac{1}{a-1} \\
\dfrac{1}{a-1} &< \left( \dfrac{a}{a-1} -1 \right)a^n \\
1 &< a^n\\
0 &< n
\end{align*}
So $f(n) \in \Theta(a^n)$
\\\\
\textbf{Case 3: 0 < a < 1}
In this case, we know that $0 < \Sigma^{n}_{i=1} a^i < 1$, so we can say that $f(n) \in \Theta(1)$
\end{document}














