% --------------------------------------------------------------
% This is all preamble stuff that you don't have to worry about.
% Head down to where it says "Start here"
% --------------------------------------------------------------

\documentclass[12pt]{article}

\usepackage[margin=1in]{geometry}
\usepackage{amsmath,amsthm,amssymb}
\usepackage{enumerate}
\usepackage{graphicx}
\usepackage[english]{babel}
\usepackage[utf8x]{inputenc}
\usepackage[T1]{fontenc}
\usepackage{enumitem}
\usepackage{fancyhdr}
\pagestyle{fancy}


\newcommand{\N}{\mathbb{N}}
\newcommand{\Z}{\mathbb{Z}}
\newcommand{\R}{\mathbb{R}}
\newcommand{\Q}{\mathbb{Q}}
\newcommand{\C}{\mathbb{C}}

\newenvironment{theorem}[2][Theorem]{\begin{trivlist}
\item[\hskip \labelsep {\bfseries #1}\hskip \labelsep {\bfseries #2.}]}{\end{trivlist}}
\newenvironment{lemma}[2][Lemma]{\begin{trivlist}
\item[\hskip \labelsep {\bfseries #1}\hskip \labelsep {\bfseries #2.}]}{\end{trivlist}}
\newenvironment{exercise}[2][Exercise]{\begin{trivlist}
\item[\hskip \labelsep {\bfseries #1}\hskip \labelsep {\bfseries #2.}]}{\end{trivlist}}
\newenvironment{problem}[2][Problem]{\begin{trivlist}
\item[\hskip \labelsep {\bfseries #1}\hskip \labelsep {\bfseries #2.}]}{\end{trivlist}}
\newenvironment{question}[2][Question]{\begin{trivlist}
\item[\hskip \labelsep {\bfseries #1}\hskip \labelsep {\bfseries #2.}]}{\end{trivlist}}
\newenvironment{corollary}[2][Corollary]{\begin{trivlist}
\item[\hskip \labelsep {\bfseries #1}\hskip \labelsep {\bfseries #2.}]}{\end{trivlist}}

\lhead{Homework 2, October 12, 2022}
\rhead{Nicholas Tee}

\begin{document}

\section*{Question 1}
\subsection*{Part A}
16KB cache = 16,384 Bytes = $2^{14}$ Bytes $\Rightarrow$ 32 - 14 = 18 tag bits\\
64B blocks = $2^{6}$ Byte blocks $\Rightarrow$ 6 offset bits\\
14 - 6 = 8 index bits\\
0x00ffb048 = 0000 0000 1111 1111 1011 0000 0010 1000
\begin{table}[!h]
\centering
\begin{tabular}{|l|l|l|}
\hline
tag & index & offset \\ \hline
0000 0000 1111 1111 01 & 11 0000 00 & 10 1000 \\ \hline
\end{tabular}
\end{table}
\subsection*{Part B}
512KB cache = 524,288 Bytes = $2^{19}$ Bytes\\
16B blocks = $2^{4}$ byte blocks\\
8 way associative $\Rightarrow$ $2^3$\\
\textit{Index:} $19 - 3 = 16$ index bits\\
\textit{Offset:} $4= 4$ offset bits\\
\textit{Tag:} $32 - 15 - 3 = 12$ tag bits\\
0x1a2b3c4d = 0001 1010 0010 1011 0011 1100 0100 1101\\
\begin{table}[!h]
\centering
\begin{tabular}{|l|l|l|}
\hline
tag & index & offset \\ \hline
0001 1010 0010 & 1011 0011 1100 0100 & 1101 \\ \hline
\end{tabular}
\end{table}
\subsection*{Part C}
hit time = 4 cycles\\
miss time = 28 cycles\\
miss rate = 1 - hit rate = 1 - 0.75 = 0.25\\
miss penalty = miss time - hit time = 28 - 4 = 24 cycles\\
\[ AMAT = 4 + 0.25 \cdot 24 = 10 \]
AMAT = 10 cycles
\newpage
\subsection*{Part D}
L1 hit time = 2 cycles\\
L1 hit rate = 0.75 $\Rightarrow$ 0.25 miss rate\\
L2 hit time = 2 + 8 = 10 cycles\\
L2 hit rate = 0.5 $\Rightarrow$ 0.5 miss rate\\
L2 miss penalty = 80 cycles
\begin{align*}
AMAT &= 2 + 0.25 \cdot ((10 + 0.5 \cdot 80) - 2)\\
&= 2 + 0.25 \cdot ((50) - 2)\\
&= 2 + 0.25 \cdot (48)\\
&= 2 + 12 = 14
\end{align*}
AMAT = 14
\section*{Question 2}
16KB capacity\\
16B block size/cache line\\
2 way associativity\\
no. of blocks = $2^{14} / 2^{4} = 2 ^{10} = 1024$ blocks, with two longs each and there are $512$ sets.\\\\
\textbf{a)}Since each block can hold up to two longs, then the cache should miss every other long. Since if we look for a[0][0] in the cache, it will miss, and then both a[0][0] and a[0][1] will be stored in the cache.\\\\
\textbf{Cache hit rate:} $50\%$\\\\
\textbf{b)}Due to the way that the loops access the array, the hit rate will be $0\%$. Since the code accesses the array column wise, there will not be enough space in the cache for a hit\\\\
\textbf{c)}Compulsory miss, due to a lack in block size\\\\
\textbf{d)}Conflict miss, due to a lack of associativity. \\\\
\textbf{e)}Block size, this way more longs can be stored in the cache at a miss.\\\\
\textbf{f)}Increase associativity
\end{document}






















