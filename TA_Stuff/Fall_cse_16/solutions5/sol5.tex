% --------------------------------------------------------------
% This is all preamble stuff that you don't have to worry about.
% Head down to where it says "Start here"
% --------------------------------------------------------------

\documentclass[12pt]{article}

\usepackage[margin=1in]{geometry}
\usepackage{amsmath,amsthm,amssymb}
\usepackage{enumerate}
\usepackage{graphicx}
\usepackage[english]{babel}
\usepackage[utf8x]{inputenc}
\usepackage[T1]{fontenc}
\usepackage{enumitem}
\usepackage{fancyhdr}
\pagestyle{fancy}


\newcommand{\N}{\mathbb{N}}
\newcommand{\Z}{\mathbb{Z}}
\newcommand{\R}{\mathbb{R}}
\newcommand{\Q}{\mathbb{Q}}
\newcommand{\C}{\mathbb{C}}

\newenvironment{theorem}[2][Theorem]{\begin{trivlist}
\item[\hskip \labelsep {\bfseries #1}\hskip \labelsep {\bfseries #2.}]}{\end{trivlist}}
\newenvironment{lemma}[2][Lemma]{\begin{trivlist}
\item[\hskip \labelsep {\bfseries #1}\hskip \labelsep {\bfseries #2.}]}{\end{trivlist}}
\newenvironment{exercise}[2][Exercise]{\begin{trivlist}
\item[\hskip \labelsep {\bfseries #1}\hskip \labelsep {\bfseries #2.}]}{\end{trivlist}}
\newenvironment{problem}[2][Problem]{\begin{trivlist}
\item[\hskip \labelsep {\bfseries #1}\hskip \labelsep {\bfseries #2.}]}{\end{trivlist}}
\newenvironment{question}[2][Question]{\begin{trivlist}
\item[\hskip \labelsep {\bfseries #1}\hskip \labelsep {\bfseries #2.}]}{\end{trivlist}}
\newenvironment{corollary}[2][Corollary]{\begin{trivlist}
\item[\hskip \labelsep {\bfseries #1}\hskip \labelsep {\bfseries #2.}]}{\end{trivlist}}

\lhead{Homework 5 solutions}
\rhead{}

\begin{document}

\subsection*{Problem 5.01}
The following claim can be disproved using a counter-example. Let $n = 6$ and $k = 2$. Since $n$ is even, and $2 \leq k \leq n-2$, the preconditions have been satisfied. If we try and compute it we get:
\begin{align*}
6 \choose{2} &= \dfrac{6!}{4! \cdot 2!}\\
&= \dfrac{6 \cdot 5 \cdot 4 \cdot 3 \cdot 2}{4 \cdot 3 \cdot 2 \cdot 2}\\
&= \dfrac{6 \cdot 5}{2}\\
&= 3 \cdot 5\\
&= 15
\end{align*}
Thus we have disproved the statement since $n \choose k$ is not even when $n = 6$ and $k = 2$

\subsection*{Problem 5.02}
The claim can be proven using a direct proof. We need to show that $n^2 = 2 {n \choose 2} +n$. We have:
\begin{align*}
2 {n \choose 2} +n &= 2 \cdot \dfrac{n!}{(n-2)!2!} + n\\
&= 2 \cdot \dfrac{n(n-1)(n-2)!}{(n-2)!2} + n\\
&= n(n-1) + n\\
&= n^2 - n + n\\
&= n^2
\end{align*}
Note that this is only possible if $n \geq 2$ since if $n < 2$ then the cancellations made in this proof would not be possible, since we have to deal with negative factorials.


\newpage
\subsection*{Problem 5.03Y}
The following claim can be disproved using a counter-example. Let us use the following parameters
\begin{verbatim}
studnts = [0]    begin_s = 0    end_s = 0    target = 0
\end{verbatim}
These conditions satisfies the preconditions as $0 \leq begin\_s \leq end\_s$ and the $target$ is within the range. Since $low := begin\_s$ and $high := end\_s$ and by line $5$ of the code we can see that the condition of the while loop states that $low < high$. Now, since $0 < 0$ is not true, the program never enters the while loop and returns $-1$. Thus the statement is not true.


\subsection*{Problem 5.03N}
Assume some arbitrary array of $studnts$ and some range $begin\_s > 0$ and $end\_s < studnts.length$ such that the $target$ is not within the range of $[begin\_s - end\_s]$. This leaves us with two different cases. Recall that by line 6 of the code that $mid = \dfrac{high + low}{2}$\\\\
\textbf{Case 1: target < students[mid]} By line 8 of the code that $high = mid$. Thus, as the program iterates through the while loop, that the value of $high$ will slowly start converging towards the value of $low$, then we can say that the $[low - high]$ range will never contain $target$.\\\\
\textbf{Case 2: target > studnts[mid]} by line 10 of the code that $low = mid$. Thus as the program iterates through the while loop, that the value of $low$ will slowly start converging towards the value of $high$ then we can say that the $[low - high]$ range will never contain $target$.\\
We can also use the proposition 5.03P, it states that for any $n, k \in \Z^{+}$ that $n < 2^k$ is true. We can rewrite this statement to look like $\dfrac{n}{2^k} < 1$. We can say that $n$ is the size of the range $[begin\_s - end\_s]$, and that $k$ is the number of iterations we go throughout the function. From here we can see that each iteration, since either $high$ or $low$, gets replaced with the value of $mid$ that the range eventually converges into less than 1 element. So given the two cases and this fact, we can say that if the $target$ is not within the range, that when the function does return, it will return -1


\subsection*{Problem 5.03T}
We can use the array $[0, 1]$. We can have $low = 0$ and $high = 1$ as our starting values, and our $target = 1$. We have it such that $\dfrac{low + high}{2} = \dfrac{0 + 1}{2} = \dfrac{1}{2}$. Since, we are flooring the function, we have it such that $mid = 0$ and that $studnts[mid] = 0$. Since $0 < 1$ based on line 10 of the code, we replace low with mid. So since, low stays the same per iteration, and it is always true that $low < high$, we stay in the while loop forever, and the function is never terminated.






\end{document}















