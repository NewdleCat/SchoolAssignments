% --------------------------------------------------------------
% This is all preamble stuff that you don't have to worry about.
% Head down to where it says "Start here"
% --------------------------------------------------------------

\documentclass[12pt]{article}

\usepackage[margin=1in]{geometry}
\usepackage{amsmath,amsthm,amssymb}
\usepackage{enumerate}
\usepackage{graphicx}
\usepackage[english]{babel}
\usepackage[utf8x]{inputenc}
\usepackage[T1]{fontenc}
\usepackage{enumitem}
\usepackage{fancyhdr}
\pagestyle{fancy}


\newcommand{\N}{\mathbb{N}}
\newcommand{\Z}{\mathbb{Z}}
\newcommand{\R}{\mathbb{R}}
\newcommand{\Q}{\mathbb{Q}}
\newcommand{\C}{\mathbb{C}}

\newenvironment{theorem}[2][Theorem]{\begin{trivlist}
\item[\hskip \labelsep {\bfseries #1}\hskip \labelsep {\bfseries #2.}]}{\end{trivlist}}
\newenvironment{lemma}[2][Lemma]{\begin{trivlist}
\item[\hskip \labelsep {\bfseries #1}\hskip \labelsep {\bfseries #2.}]}{\end{trivlist}}
\newenvironment{exercise}[2][Exercise]{\begin{trivlist}
\item[\hskip \labelsep {\bfseries #1}\hskip \labelsep {\bfseries #2.}]}{\end{trivlist}}
\newenvironment{problem}[2][Problem]{\begin{trivlist}
\item[\hskip \labelsep {\bfseries #1}\hskip \labelsep {\bfseries #2.}]}{\end{trivlist}}
\newenvironment{question}[2][Question]{\begin{trivlist}
\item[\hskip \labelsep {\bfseries #1}\hskip \labelsep {\bfseries #2.}]}{\end{trivlist}}
\newenvironment{corollary}[2][Corollary]{\begin{trivlist}
\item[\hskip \labelsep {\bfseries #1}\hskip \labelsep {\bfseries #2.}]}{\end{trivlist}}

\lhead{Homework 2, April 16, 2021}
\rhead{Nicholas Tee}

\begin{document}
worked with: Brooke Zhang
\subsection*{Problem 1}
For any $x \in A$. we can say 
\[ 0x = x0 \]
Through the properties of 0
\[ 0x = x(0 + 0) \]
Using distribution
\[ 0x = 0x + 0x \]
Plus, with additive inverses we can get 
\[ 0x + (-0x) = (0x + 0x) + (-0x) \]
By associativity we get
\[ 0x + (-0x) = 0x + (0x + (-0x)) \]
\[ 0 = 0x + 0 \]
So this means that $0 = 0x$ for all $x \in A$
\subsection*{Problem 2}
Since we know that $x \neq 0$. This means that $x$ has an inverse, and that $x^{-1}x = xx^{-1}x = 1$. So we can say that.
\begin{align*}
xy &= xz\\
x^{-1}(xy) &= x^{-1}(xz)\\
(x^{-1}x)y &= (x^{-1}x)z\\
1 \cdot y &= 1 \cdot z\\
 y &= z
\end{align*}
This shows that $y = z$
\subsection*{Problem 3}
\textbf{a) } $\{\bar{ 1 }, \bar{ 2 }\}$\\
\textbf{b) } $\{\bar{ 1 }, \bar{ 3 }\}$\\
\textbf{c) } $ \{ (\bar{1}, \bar{1}) \} $\\
\textbf{d) } $ \{ \bar{1}, \bar{5}, \bar{7}, \bar{11} \} $\\
\textbf{e) } $\{ (\bar{1}, \bar{1}), (\bar{1}, \bar{3}), (\bar{2}, \bar{1}), (\bar{2}, \bar{3}) \}$\\
\textbf{f) } $\{ (\bar{1}, \bar{1}, \bar{1}), (\bar{1}, \bar{1}, \bar{2})  \}$\\
\subsection*{Problem 4(A)}
The first difference between addition and multiplication is that addition only happens once, whereas multiplication is repeated addition. for example, 1 + 3 = 4. But then 1 x 3 is = 1 + 1 + 1 = 3. Another difference is the use of negative signs. For example if you were to do 2 - 2, you would get 0. But if you multiply 2 x -2, you get -4. The most obvious and third difference is the different sign that multiplication uses. Addition uses the regular + sign whereas multiplication can use an x symbol, a dot or just parenthesis together.
\subsection*{Problem 5}
The two rings $\Z/10\Z$ and $\Z/5\Z$ are abelian under addition. We can say that for the map $\varphi: \Z/10\Z \rightarrow \Z/5\Z$, that $\varphi(1) = 2$. With that we know that there is a homomorphism between them as groups. However, since we are using $\varphi(1) = 2$, this means that $\varphi(1) \neq 1$ which can not make this a ring homomorphism. 

\subsection*{Problem 6(B)}
$ev_a$ is a one-to-one function just like our regular pre-calculus functions. Also , when you do addition between two functions you only get one value. In regular function you most likely would get a numerical value, whereas in rings you would get whatever A is. The difference between them is that you would use $ev_a$ for only polynomials, whereas in earlier mathematics we did these types of operations on any functions, whether it be polynomials or linear functions.


\end{document}